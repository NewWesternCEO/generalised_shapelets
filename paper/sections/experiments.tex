\section{Experiments}
Our generalised shapelet transform, contrasted with the classical shapelet transform, has two extra degrees of freedom: the choice of interpolation scheme $\iota$, and the choice of discrepancy function $\pi^A_S$. In our experiments, we consider $\pi^A_S$ given by either of equations \eqref{eq:learnt-discrepancy} or \eqref{eq:logsignature-discrepancy}. Meanwhile, we take $\iota$ to be piecewise linear interpolation, because efficient algorithms for computing the logsignature transform only exist for piecewise linear paths \cite{signatory}.

TODO: we need to consider more disrepancy functions than this.

\subsection{The UCR (Univariate) Time-Series Archive}

\begin{table}[ht]
    \caption{Scores for four different metrics on a subset of the UCR time-series archive.}
    \label{tab:ucr_results}
    \centering
    \begin{tabular}{lccc}
\toprule
&  \multicolumn{3}{c}{Discrepancy} \\
\textbf{Dataset} &    L2 &   old &  logsig-3-diagonal \\
\midrule
ArrowHead                & 0.720 &     - &                  - \\
BME                      & 0.440 & 0.593 &              0.393 \\
CBF                      & 0.351 & 0.898 &              0.353 \\
Chinatown                & 0.595 & 0.697 &              0.918 \\
Coffee                   & 0.893 & 0.929 &              0.643 \\
ECG200                   & 0.740 & 0.660 &              0.640 \\
ECGFiveDays              & 0.502 & 0.843 &              0.654 \\
FreezerSmallTrain        & 0.656 & 0.796 &              0.751 \\
GunPoint                 & 0.573 & 0.433 &              0.567 \\
GunPointAgeSpan          & 0.642 & 0.509 &              0.684 \\
GunPointMaleVersusFemale & 0.525 & 0.991 &              0.475 \\
GunPointOldVersusYoung   & 0.997 & 0.886 &              0.927 \\
ItalyPowerDemand         & 0.867 & 0.879 &              0.501 \\
MoteStrain               & 0.876 & 0.716 &              0.588 \\
SmoothSubspace           & 0.467 & 0.900 &              0.680 \\
SonyAIBORobotSurface1    & 0.428 & 0.421 &              0.429 \\
SonyAIBORobotSurface2    & 0.800 & 0.626 &              0.605 \\
SyntheticControl         & 0.897 & 0.927 &              0.887 \\
ToeSegmentation1         & 0.526 & 0.833 &              0.526 \\
TwoLeadECG               & 0.943 & 0.910 &              0.666 \\
UMD                      & 0.486 & 0.500 &              0.458 \\
Wine                     & 0.500 & 0.426 &              0.685 \\ 
\midrule
Wins &   7 &   11 &                  4 \\
\bottomrule
\end{tabular}

\end{table}


\subsection{The UEA (Multivariate) Time-Series Archive}

\subsubsection{Comparison}
\begin{table}[ht]
    \centering
    \caption{}
    \label{tab:uea_comparison_results}
    \begin{tabular}{lccc}
\toprule
{} &  logsig-3diagonal &  L2-diagonal &   old \\
\midrule
AtrialFibrillation &             0.333 &        0.333 & 0.467 \\
BasicMotions       &             1.000 &        0.950 & 0.975 \\
ERing              &             0.574 &        0.919 & 0.737 \\
FingerMovements    &             0.450 &        0.560 & 0.490 \\
JapaneseVowels     &             0.646 &        0.914 & 0.949 \\
Libras             &             0.739 &        0.661 & 0.661 \\
NATOPS             &                 - &        0.861 &     - \\
PenDigits          &             0.967 &        0.956 & 0.956 \\
RacketSports       &             0.717 &        0.717 & 0.605 \\ 
\midrule
Wins &                 3 &            4 &    2 \\
\bottomrule
\end{tabular}

\end{table}
%
%\subsubsection{Noise}
%\begin{table}[ht]
%    \caption{}
%    \label{tab:uea_noise}
%    \centering
%    \begin{tabular}{lccc}
\toprule
& \multicolumn{3}{c}{\textbf{Discrepancy}} \\
\textbf{Dataset} &  L2-diagonal-False &   L2-diagonal-True &                old \\
\midrule
BasicMotions3     &  0.360 $\pm$ 0.175 &  0.320 $\pm$ 0.157 &  0.520 $\pm$ 0.262 \\
BasicMotions30    &  0.250 $\pm$ 0.000 &  0.280 $\pm$ 0.067 &  0.360 $\pm$ 0.213 \\
BasicMotions9     &  0.365 $\pm$ 0.139 &  0.325 $\pm$ 0.075 &  0.450 $\pm$ 0.202 \\
FingerMovements3  &  0.496 $\pm$ 0.009 &  0.536 $\pm$ 0.051 &  0.498 $\pm$ 0.054 \\
FingerMovements30 &  0.512 $\pm$ 0.029 &  0.506 $\pm$ 0.065 &  0.522 $\pm$ 0.044 \\
FingerMovements9  &  0.504 $\pm$ 0.047 &  0.510 $\pm$ 0.031 &  0.480 $\pm$ 0.041 \\
JapaneseVowels3   &  0.812 $\pm$ 0.158 &  0.725 $\pm$ 0.195 &  0.857 $\pm$ 0.027 \\
JapaneseVowels30  &  0.419 $\pm$ 0.221 &  0.209 $\pm$ 0.081 &  0.611 $\pm$ 0.175 \\
JapaneseVowels9   &  0.432 $\pm$ 0.221 &  0.556 $\pm$ 0.324 &  0.805 $\pm$ 0.041 \\ 
\midrule
Wins &                  0 &                 2 &    7 \\
\bottomrule
\end{tabular}

%\end{table}
%
%\subsubsection{Length}
%\begin{table}[ht]
%    \caption{}
%    \label{tab:uea_length}
%    \centering
%    \begin{tabular}{lcc}
\toprule
& \multicolumn{4}{c}{\textbf{Discrepancy}} \\
\textbf{Dataset} &  L2-diagonal-False &  L2-diagonal-True \\
\midrule
BasicMotions    &              0.325 &             0.925 \\
FingerMovements &              0.540 &             0.530 \\
JapaneseVowels  &              0.743 &             0.897 \\ 
\midrule
Wins &                  1 &                 2 \\
\bottomrule
\end{tabular}

%\end{table}
%
%\subsubsection{Missingness}
%\begin{table}[ht]
%    \caption{}
%    \label{tab:uea_missingness}
%    \centering
%    \begin{tabular}{lcc}
\toprule
{} &  L2-diagonal &  logsig-3-diagonal \\
\midrule
BasicMotions10   &        0.925 &                  - \\
JapaneseVowels10 &        0.859 &              0.651 \\
JapaneseVowels30 &        0.895 &              0.641 \\
JapaneseVowels50 &        0.784 &              0.632 \\ 
\midrule
Wins &            4 &                  0 \\
\bottomrule
\end{tabular}

%\end{table}
